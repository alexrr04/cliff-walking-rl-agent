\section{Comparación de los algoritmos}

En esta sección se comparan los algoritmos \textit{Value Iteration}, \textit{Direct Estimation} y \textit{Q-Learning} en el entorno. Con los mejores parámetros obtenidos en los experimentos previos, se evalúa el rendimiento de cada algoritmo.

\subsection{Diseño experimental}

El objetivo de este experimento es descubrir que algoritmo es el más eficiente para resolver el problema del entorno. Para ello, se comparan los algoritmos \textit{Value Iteration}, \textit{Direct Estimation} y \textit{Q-Learning} en el entorno.

\begin{table}[H]
    \centering
    \begin{tabularx}{\textwidth}{|p{4cm}|X|} % Especificar el ancho de las columnas
        \hline % Línea horizontal superior
        \textbf{Observación} & Los tres algoritmos tienen funcionamientos distintos, así como puntos fuertes y débiles. Uno de ellos será el más eficiente para resolver el problema del entorno.
        \\ \hline
        \textbf{Planteamiento} & Se ejecuta cada algoritmo con sus mejores parámetros obtenidos en los experimentos previos. Se evalúa la tasa de éxito, tiempo de entrenamiento, número de pasos medio y recompensa media.
        \\ \hline
        \textbf{Hipótesis} & El algoritmo \textit{Value Iteration} es el más estable y eficaz para resolver el problema, mientras que \textit{Direct Estimation} es el más ineficiente.
        \\ \hline
        \textbf{Método} & 
        \begin{itemize}
            \item Se fijan los mejores parámetros obtenidos en los experimentos previos para cada algoritmo. La única excepción es en el algoritmo \textit{Q-Learning}, donde la penalización sobre la acción izquierda se pone a -1 en vez de -50 para poder comparar las recompensas con los otros algoritmos.
            \item Se evalúa la política obtenida para cada algoritmo probándola con 500 episodios.
            \item Se repite el proceso para los algoritmos de \textit{Q-Learning} y \textit{Direct Estimation} 10 veces. Para el algoritmo \textit{Value Iteration} se hace una sola vez, ya que siempre obtiene la misma política.
        \end{itemize}
        \\ \hline
    \end{tabularx}
    \caption{Comparación de algoritmos}
    \label{tab:algorithmComparisonExp}
\end{table}

\subsection{Resultados}