\subsection{Experimentación}

\subsubsection{Experimento factor de descuento \& \(\epsilon\) decay}

\paragraph{Diseño experimental}

% TODO: Usar esta tabla para cada experimento
\begin{table}[H]
    \centering
    \begin{tabularx}{\textwidth}{|p{4cm}|X|} % Especificar el ancho de las columnas
        \hline % Línea horizontal superior
        \textbf{Observación} & El factor de descuento ($\gamma$) y la tasa de decaimiento de $\epsilon$ son parámetros críticos en el algoritmo Q-Learning. 
        \\ \hline 
        \textbf{Planteamiento} & Para cada pareja de valores de $\gamma$ y $decay$, se compara la tasa de acierto (llegar al estado final), la recompensa media, número de pasos y tiempo de entrenamiento del algoritmo.
        \\ \hline 
        \textbf{Hipótesis} & Un mayor factor de descuento y una tasa de decaimiento de $\epsilon$ más lenta mejorarán el rendimiento del algoritmo.
        \\ \hline 
        \textbf{Método} & 
        \begin{itemize}
            \item Se fijan 1000 episodios de entrenamiento, tasa de aprendizaje \(\alpha = 0.1\) y \(\epsilon\) inicial de $0.9$.
            \item Se eligen los siguientes valores para \(\gamma\) y $decay$: \(\gamma \in \{0.5, 0.7, 0.9, 0.95, 0.99\}\) y $decay \in \{0.1, 0.2, 0.5, 0.8\}$.
            \item Para cada combinación de \(\gamma\) y $decay$, se ejecuta el algoritmo Q-Learning en el entorno.
            \item Se evalúa la política obtenida probándola con 500 episodios.
            \item Se repite el proceso para cada combinación de \(\gamma\) y $decay$ 20 veces.
        \end{itemize}
        \\ \hline
    \end{tabularx}
    \caption{Q-Learning - Experimento 1 - Factor de descuento \& $\epsilon$ decay}
    \label{tab:diseñoQLEarningExp1}
\end{table}

\paragraph{Resultados}

\subsubsection{Experimento tasa de aprendizaje \& \(\epsilon\)}

\paragraph{Diseño experimental}

\begin{table}[H]
    \centering
    \begin{tabularx}{\textwidth}{|p{4cm}|X|} % Especificar el ancho de las columnas
        \hline % Línea horizontal superior
        \textbf{Observación} &  
        \\ \hline 
        \textbf{Planteamiento} & 
        \\ \hline 
        \textbf{Hipótesis} & 
        \\ \hline 
        \textbf{Método} & 
        \begin{itemize}
            \item 
        \end{itemize}
        \\ \hline
    \end{tabularx}
    \caption{Experimento 1}
    \label{tab:tabla1}
\end{table}

\paragraph{Resultados}

\subsubsection{Experimento número de episodios}

\paragraph{Diseño experimental}

\begin{table}[H]
    \centering
    \begin{tabularx}{\textwidth}{|p{4cm}|X|} % Especificar el ancho de las columnas
        \hline % Línea horizontal superior
        \textbf{Observación} &  
        \\ \hline 
        \textbf{Planteamiento} & 
        \\ \hline 
        \textbf{Hipótesis} & 
        \\ \hline 
        \textbf{Método} & 
        \begin{itemize}
            \item 
        \end{itemize}
        \\ \hline
    \end{tabularx}
    \caption{Experimento 1}
    \label{tab:tabla1}
\end{table}

\paragraph{Resultados}

\subsubsection{Experimento penalización de la acción izquierda}

\paragraph{Diseño experimental}

\begin{table}[H]
    \centering
    \begin{tabularx}{\textwidth}{|p{4cm}|X|} % Especificar el ancho de las columnas
        \hline % Línea horizontal superior
        \textbf{Observación} &  
        \\ \hline 
        \textbf{Planteamiento} & 
        \\ \hline 
        \textbf{Hipótesis} & 
        \\ \hline 
        \textbf{Método} & 
        \begin{itemize}
            \item 
        \end{itemize}
        \\ \hline
    \end{tabularx}
    \caption{Experimento 1}
    \label{tab:tabla1}
\end{table}

\paragraph{Resultados}