\subsection{Experimentación}

\subsubsection{Experimento factor de descuento \& \(\epsilon\) decay}

\paragraph{Diseño experimental}

% TODO: Usar esta tabla para cada experimento
\begin{table}[H]
    \centering
    \begin{tabularx}{\textwidth}{|p{4cm}|X|} % Especificar el ancho de las columnas
        \hline % Línea horizontal superior
        \textbf{Observación} &  
        \\ \hline 
        \textbf{Planteamiento} & 
        \\ \hline 
        \textbf{Hipótesis} & 
        \\ \hline 
        \textbf{Método} & 
        \begin{itemize}
            \item 
        \end{itemize}
        \\ \hline
    \end{tabularx}
    \caption{Experimento 1}
    \label{tab:tabla1}
\end{table}

\paragraph{Resultados}

\subsubsection{Experimento tasa de aprendizaje \& \(\epsilon\)}

\paragraph{Diseño experimental}

\begin{table}[H]
    \centering
    \begin{tabularx}{\textwidth}{|p{4cm}|X|} % Especificar el ancho de las columnas
        \hline % Línea horizontal superior
        \textbf{Observación} &  
        \\ \hline 
        \textbf{Planteamiento} & 
        \\ \hline 
        \textbf{Hipótesis} & 
        \\ \hline 
        \textbf{Método} & 
        \begin{itemize}
            \item 
        \end{itemize}
        \\ \hline
    \end{tabularx}
    \caption{Experimento 1}
    \label{tab:tabla1}
\end{table}

\paragraph{Resultados}

\subsubsection{Experimento número de episodios}

\paragraph{Diseño experimental}

\begin{table}[H]
    \centering
    \begin{tabularx}{\textwidth}{|p{4cm}|X|} % Especificar el ancho de las columnas
        \hline % Línea horizontal superior
        \textbf{Observación} &  
        \\ \hline 
        \textbf{Planteamiento} & 
        \\ \hline 
        \textbf{Hipótesis} & 
        \\ \hline 
        \textbf{Método} & 
        \begin{itemize}
            \item 
        \end{itemize}
        \\ \hline
    \end{tabularx}
    \caption{Experimento 1}
    \label{tab:tabla1}
\end{table}

\paragraph{Resultados}

\subsubsection{Experimento penalización de la acción izquierda}

\paragraph{Diseño experimental}

\begin{table}[H]
    \centering
    \begin{tabularx}{\textwidth}{|p{4cm}|X|} % Especificar el ancho de las columnas
        \hline % Línea horizontal superior
        \textbf{Observación} &  
        \\ \hline 
        \textbf{Planteamiento} & 
        \\ \hline 
        \textbf{Hipótesis} & 
        \\ \hline 
        \textbf{Método} & 
        \begin{itemize}
            \item 
        \end{itemize}
        \\ \hline
    \end{tabularx}
    \caption{Experimento 1}
    \label{tab:tabla1}
\end{table}

\paragraph{Resultados}